\listfiles
\documentclass[12pt]{article}


% Basic Packages

\usepackage{lineno}
\usepackage{hyperref}
\usepackage{graphicx}
\usepackage{subfig}
\usepackage{float}


% Margins

\usepackage[a4paper,
bindingoffset=0.2in,
left=1in,
right=1in,
top=1.2in,
bottom=1.2in,
footskip=.25in]{geometry}

\linespread{2}
\linenumbers


% Title


\title{\renewcommand{\baselinestretch}{1.17}\large\bf Beyond Microsoft}

% Authors

\author{\normalsize Dylan Padilla}


% Begin Document

\begin{document}

\date{}

\maketitle

\noindent
School of Life Sciences, Arizona State University \\
Email: \url{dpadil10@asu.edu} \vspace{6px}


% Abstract

\noindent
{\normalsize{\bf Abstract}
\newline

This is a live demo to demonstrate how to write a paper in LaTEX. It is part of the series of YouTube shorts that the ASN Graduate Council is making.
}

\noindent
{\small{\bf Keywords:}{free software, ASN, YouTube}
}


\newpage
\noindent
\section*{\normalsize Introduction}
``Free Software'' means software that respects users' freedom and community. Roughly, it means that the users have the freedom to run, copy, study, and improve the software. Thus, free software is a matter of liberty, not price. To understand the concept, please watch the series of YouTube shorts that the ASN graduate council is making.


\vspace{30pt}

\begin{figure}[!htb]
\centering
\includegraphics[scale=0.5]{imgs/ASN_logo.png}
\caption{Caption here ...}
\label{fig:1}
\end{figure}



\end{document}























































